\textbf{CREX Table C} \emph{-- Data description operators}

\begin{longtable}[]{@{}llll@{}}
\toprule
REFERENCE & OPERAND & OPERATOR NAME & OPERATION DEFINITION\tabularnewline
\midrule
\endhead
C 01 & YYY & Data width replacement & YYY characters (from 000 to 999) replace specified Table B data width\tabularnewline
C 02 & YYY & Scale factor replacement & YYY (from --99 to 999) replaces the specified Table B scale factor\tabularnewline
C 05 & YYY & Character insertion & YYY characters (from 001 to 999), including spaces, are inserted as a data field\tabularnewline
\begin{minipage}[t]{0.22\columnwidth}\raggedright
C 07\strut
\end{minipage} & \begin{minipage}[t]{0.22\columnwidth}\raggedright
YYY\strut
\end{minipage} & \begin{minipage}[t]{0.22\columnwidth}\raggedright
Units replacement\strut
\end{minipage} & \begin{minipage}[t]{0.22\columnwidth}\raggedright
Change unit to unit defined in Common Code table C--6 by code figure equal to YYY, for example:

YYY = 040 changes unit to Celsius

YYY = 741 changes unit to km h\textsuperscript{--1}

YYY = 201 changes unit to knot

YYY = 740 changes unit to km\strut
\end{minipage}\tabularnewline
\begin{minipage}[t]{0.22\columnwidth}\raggedright
C 41\strut
\end{minipage} & \begin{minipage}[t]{0.22\columnwidth}\raggedright
000\strut
\end{minipage} & \begin{minipage}[t]{0.22\columnwidth}\raggedright
Define event\strut
\end{minipage} & \begin{minipage}[t]{0.22\columnwidth}\raggedright
This operator denotes the beginning of the definition of an event

(see Note 2)\strut
\end{minipage}\tabularnewline
C 41 & 999 & Cancel define event & This operator denotes the conclusion of the event definition that was begun via the previous C 41 000 operator\tabularnewline
C 42 & 000 & Define conditioning event & This operator denotes the beginning of the definition of a conditioning event (see Note 2)\tabularnewline
C 42 & 999 & \vtop{\hbox{\strut Cancel define}\hbox{\strut conditioning event}} & This operator denotes the conclusion of the conditioning event definition that was begun via the previous C 42 000 operator\tabularnewline
C 43 & 000 & \vtop{\hbox{\strut Categorical forecast}\hbox{\strut values follow}} & The values which follow are categorical forecast values (see Note 3)\tabularnewline
C 43 & 999 & \vtop{\hbox{\strut Cancel categorical}\hbox{\strut forecast values follow}} & This operator denotes the conclusion of the definition of categorical forecast values that was begun via the previous C 43 000 operator\tabularnewline
C 60 & YYY & National letters insertion (see Note 4) & YYY national letters including spaces are inserted as a data field\tabularnewline
\bottomrule
\end{longtable}

Notes:

(1) The operations specified by operator descriptors C 41 000, C 42 000 and C 43 000 remain defined until cancelled or until the end of the data subset. Regulation 95.3.4.2 shall not apply here.

(2) An event, as defined for use with operators C 41 000 and C 42 000, is a set of one or more circumstances described using appropriate Table B descriptors along with their corresponding data values. The grouping of such descriptors together as a single "event" allows them to be collectively assigned as the target of a separate descriptor such as B 33 045 or B 33 046. When defining a circumstance within an event, descriptor B 33 042 may be employed preceding the appropriate Table B descriptor in order to indicate that the corresponding value is actually a bound for a range of values.

\emph{(continued)}

\emph{\\
(CREX Table C -- continued)}

(3) A categorical forecast value represents a "best guess" from among a set of related, and often mutually exclusive, data values or categories. Operator C 43 000 may be used to designate one or more values as being categorical forecast values, and descriptor B 33 042 may be employed preceding any such value in order to indicate that that value is actually a bound for a range of values.

(4) Only the characters from the International Telegraphic Alphabet No. 2 (ITA2) are likely to be transmitted accurately to all recipients.
