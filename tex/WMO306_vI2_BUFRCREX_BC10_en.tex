\textbf{B/C10} -- \textbf{Regulations for reporting SHIP data in TDCF}

\textbf{TM~308009} -- \textbf{BUFR template for synoptic reports from sea stations suitable for SHIP data}

\begin{longtable}[]{@{}lll@{}}
\toprule
\endhead
& & \textbf{Sequence for representation of synoptic reports from a sea station suitable for SHIP data}\tabularnewline
\textbf{3 08 009} & \textbf{3 01 093} & \textbf{Ship identification, movement, date/time, horizontal and vertical coordinates}\tabularnewline
& \textbf{3 02 001} & \textbf{Pressure and 3-hour pressure change}\tabularnewline
& \textbf{3 02 054} & \textbf{Ship ``instantaneous'' data}\tabularnewline
& \textbf{0 08 002} & \textbf{Vertical significance (surface observations)}\tabularnewline
& \textbf{3 02 055} & \textbf{Icing and ice}\tabularnewline
& \textbf{3 02 057} & \textbf{Ship marine data}\tabularnewline
& \textbf{3 02 060} & \textbf{Ship ``period'' data}\tabularnewline
\bottomrule
\end{longtable}

This BUFR template for synoptic reports from sea stations further expands as follows:

\begin{longtable}[]{@{}lllll@{}}
\toprule
\endhead
& & & & Unit, scale\tabularnewline
& & & \textbf{Ship identification, movement, date/time, horizontal and vertical coordinates} &\tabularnewline
\textbf{3 01 093} & 3 01 036 & 0 01 011 & Ship or mobile land station identifier D\ldots.D & CCITT IA5, 0\tabularnewline
& & 0 01 012 & Direction of motion of moving observing platform (see Note 1) D\textsubscript{s} & Degree true, 0\tabularnewline
& & 0 01 013 & \vtop{\hbox{\strut Speed of motion of moving observing platform}\hbox{\strut (see Note 2) v\textsubscript{s}}} & m s\textsuperscript{--1}, 0\tabularnewline
& & 0 02 001 & Type of station i\textsubscript{x} & Code table, 0\tabularnewline
& & 0 04 001 & Year & Year, 0\tabularnewline
& & 0 04 002 & Month & Month, 0\tabularnewline
& & 0 04 003 & Day YY & Day, 0\tabularnewline
& & 0 04 004 & Hour GG & Hour, 0\tabularnewline
& & 0 04 005 & Minute gg & Minute, 0\tabularnewline
& & 0 05 002 & Latitude (coarse accuracy) L\textsubscript{a}L\textsubscript{a}L\textsubscript{a} & Degree, 2\tabularnewline
& & 0 06 002 & Longitude (coarse accuracy) L\textsubscript{o}L\textsubscript{o}L\textsubscript{o}L\textsubscript{o} & Degree, 2\tabularnewline
& 0 07 030 & & Height of station ground (platform) above mean sea level & m, 1\tabularnewline
& 0 07 031 & & Height of barometer above mean sea level & m, 1\tabularnewline
& & & Pressure and 3-hour pressure change &\tabularnewline
\textbf{3 02 001} & 0 10 004 & & Pressure P\textsubscript{0}P\textsubscript{0}P\textsubscript{0}P\textsubscript{0} & Pa, --1\tabularnewline
& 0 10 051 & & Pressure reduced to mean sea level PPPP & Pa, --1\tabularnewline
& 0 10 061 & & 3-hour pressure change ppp & Pa, --1\tabularnewline
& 0 10 063 & & Characteristic of pressure tendency a & Code table, 0\tabularnewline
& & & \textbf{Ship ``instantaneous'' data} &\tabularnewline
& & & \emph{Ship temperature and humidity data} &\tabularnewline
\textbf{3 02 054} & \textbf{3 02 052} & 0 07 032 & \vtop{\hbox{\strut Height of sensor above local ground (or deck of marine platform)}\hbox{\strut (for temperature and humidity measurement)}} & m, 2\tabularnewline
\begin{minipage}[t]{0.17\columnwidth}\raggedright
\strut
\end{minipage} & \begin{minipage}[t]{0.17\columnwidth}\raggedright
\strut
\end{minipage} & \begin{minipage}[t]{0.17\columnwidth}\raggedright
0 07 033\strut
\end{minipage} & \begin{minipage}[t]{0.17\columnwidth}\raggedright
Height of sensor above water surface

(for temperature and humidity measurement)\strut
\end{minipage} & \begin{minipage}[t]{0.17\columnwidth}\raggedright
m, 1\strut
\end{minipage}\tabularnewline
& & 0 12 101 & Temperature/air temperature s\textsubscript{n}TTT & K, 2\tabularnewline
& & 0 02 039 & Method of wet-bulb temperature measurement & Code table, 0\tabularnewline
& & 0 12 102 & Wet-bulb temperature s\textsubscript{w}T\textsubscript{b}T\textsubscript{b}T\textsubscript{b} & K, 2\tabularnewline
& & 0 12 103 & Dewpoint temperature s\textsubscript{n}T\textsubscript{d}T\textsubscript{d}T\textsubscript{d} & K, 2\tabularnewline
& & 0 13 003 & Relative humidity & \%, 0\tabularnewline
& & & \emph{Ship visibility data} &\tabularnewline
\begin{minipage}[t]{0.17\columnwidth}\raggedright
\strut
\end{minipage} & \begin{minipage}[t]{0.17\columnwidth}\raggedright
\hypertarget{section}{%
\subsection{3 02 053}\label{section}}\strut
\end{minipage} & \begin{minipage}[t]{0.17\columnwidth}\raggedright
\hypertarget{section-1}{%
\subsection{0 07 032}\label{section-1}}\strut
\end{minipage} & \begin{minipage}[t]{0.17\columnwidth}\raggedright
Height of sensor above local ground (or deck of marine platform)\\
(for visibility measurement)\strut
\end{minipage} & \begin{minipage}[t]{0.17\columnwidth}\raggedright
m, 2\strut
\end{minipage}\tabularnewline
\begin{minipage}[t]{0.17\columnwidth}\raggedright
\strut
\end{minipage} & \begin{minipage}[t]{0.17\columnwidth}\raggedright
\strut
\end{minipage} & \begin{minipage}[t]{0.17\columnwidth}\raggedright
0 07 033\strut
\end{minipage} & \begin{minipage}[t]{0.17\columnwidth}\raggedright
Height of sensor above water surface

(for visibility measurement)\strut
\end{minipage} & \begin{minipage}[t]{0.17\columnwidth}\raggedright
m, 1\strut
\end{minipage}\tabularnewline
& & 0 20 001 & Horizontal visibility VV & m, --1\tabularnewline
& 0 07 033 & & \vtop{\hbox{\strut Height of sensor above water surface}\hbox{\strut (set to missing to cancel the previous value)}} & m, 1\tabularnewline
& & & \emph{Precipitation past 24 hours} &\tabularnewline
\begin{minipage}[t]{0.17\columnwidth}\raggedright
\strut
\end{minipage} & \begin{minipage}[t]{0.17\columnwidth}\raggedright
\textbf{3 02 034}\strut
\end{minipage} & \begin{minipage}[t]{0.17\columnwidth}\raggedright
0 07 032\strut
\end{minipage} & \begin{minipage}[t]{0.17\columnwidth}\raggedright
\hypertarget{height-of-sensor-above-local-ground-or-deck-of-marine-platform-for-precipitation-measurement}{%
\subparagraph{\texorpdfstring{Height of sensor above local ground (or deck of marine platform)\\
(for precipitation measurement)}{Height of sensor above local ground (or deck of marine platform) (for precipitation measurement)}}\label{height-of-sensor-above-local-ground-or-deck-of-marine-platform-for-precipitation-measurement}}\strut
\end{minipage} & \begin{minipage}[t]{0.17\columnwidth}\raggedright
m, 2\strut
\end{minipage}\tabularnewline
& & 0 13 023 & Total precipitation past 24 hours R\textsubscript{24}R\textsubscript{24}R\textsubscript{24}R\textsubscript{24} & kg m\textsuperscript{--2}, 1\tabularnewline
\begin{minipage}[t]{0.17\columnwidth}\raggedright
\strut
\end{minipage} & \begin{minipage}[t]{0.17\columnwidth}\raggedright
0 07 032\strut
\end{minipage} & \begin{minipage}[t]{0.17\columnwidth}\raggedright
\strut
\end{minipage} & \begin{minipage}[t]{0.17\columnwidth}\raggedright
\hypertarget{height-of-sensor-above-local-ground-or-deck-of-marine-platform-set-to-missing-to-cancel-the-previous-value}{%
\subparagraph{\texorpdfstring{Height of sensor above local ground (or deck of marine platform)\\
(set to missing to cancel the previous value)}{Height of sensor above local ground (or deck of marine platform) (set to missing to cancel the previous value)}}\label{height-of-sensor-above-local-ground-or-deck-of-marine-platform-set-to-missing-to-cancel-the-previous-value}}\strut
\end{minipage} & \begin{minipage}[t]{0.17\columnwidth}\raggedright
m, 2\strut
\end{minipage}\tabularnewline
& & & \emph{General cloud information} &\tabularnewline
& 3 02 004 & 0 20 010 & Cloud cover (total) N & \%, 0\tabularnewline
& & 0 08 002 & Vertical significance \textbf{(surface observations)} & Code table, 0\tabularnewline
& & 0 20 011 & Cloud amount (of low or middle clouds) N\textsubscript{h} & Code table, 0\tabularnewline
& & 0 20 013 & Height of base of cloud h & m, --1\tabularnewline
& & 0 20 012 & Cloud type (low clouds) C\textsubscript{L} & Code table, 0\tabularnewline
& & 0 20 012 & Cloud type (middle clouds) C\textsubscript{M} & Code table, 0\tabularnewline
& & 0 20 012 & Cloud type (high clouds) C\textsubscript{H} & Code table, 0\tabularnewline
& 1 01 000 & & Delayed replication of 1 descriptor &\tabularnewline
& 0 31~001 & & Delayed descriptor replication factor & Numeric, 0\tabularnewline
& 3 02 005 & 0 08 002 & Vertical significance \textbf{(surface observations)} & Code table, 0\tabularnewline
& & 0 20 011 & Cloud amount N\textsubscript{s} & Code table, 0\tabularnewline
& & 0 20 012 & Cloud type C & Code table, 0\tabularnewline
& & 0 20 013 & Height of base of cloud h\textsubscript{s}h\textsubscript{s} & m, --1\tabularnewline
\begin{minipage}[t]{0.17\columnwidth}\raggedright
\textbf{0 08 002}\strut
\end{minipage} & \begin{minipage}[t]{0.17\columnwidth}\raggedright
\strut
\end{minipage} & \begin{minipage}[t]{0.17\columnwidth}\raggedright
\strut
\end{minipage} & \begin{minipage}[t]{0.17\columnwidth}\raggedright
Vertical significance \textbf{(surface observations)}

(set to missing to cancel the previous value)\strut
\end{minipage} & \begin{minipage}[t]{0.17\columnwidth}\raggedright
Code table, 0\strut
\end{minipage}\tabularnewline
& & & \textbf{Icing and ice} &\tabularnewline
\begin{minipage}[t]{0.17\columnwidth}\raggedright
\textbf{3 02 055}\strut
\end{minipage} & \begin{minipage}[t]{0.17\columnwidth}\raggedright
0 20 031\strut
\end{minipage} & \begin{minipage}[t]{0.17\columnwidth}\raggedright
\hypertarget{section-2}{%
\subsection{}\label{section-2}}\strut
\end{minipage} & \begin{minipage}[t]{0.17\columnwidth}\raggedright
Ice deposit (thickness) E\textsubscript{s}E\textsubscript{s}\strut
\end{minipage} & \begin{minipage}[t]{0.17\columnwidth}\raggedright
m, 2\strut
\end{minipage}\tabularnewline
\begin{minipage}[t]{0.17\columnwidth}\raggedright
\strut
\end{minipage} & \begin{minipage}[t]{0.17\columnwidth}\raggedright
0 20 032\strut
\end{minipage} & \begin{minipage}[t]{0.17\columnwidth}\raggedright
\hypertarget{section-3}{%
\subsection{}\label{section-3}}\strut
\end{minipage} & \begin{minipage}[t]{0.17\columnwidth}\raggedright
Rate of ice accretion R\textsubscript{s}\strut
\end{minipage} & \begin{minipage}[t]{0.17\columnwidth}\raggedright
Code table, 0\strut
\end{minipage}\tabularnewline
& 0 20 033 & & Cause of ice accretion I\textsubscript{s} & Flag table, 0\tabularnewline
& 0 20 034 & & Sea-ice concentration c\textsubscript{i} & Code table, 0\tabularnewline
& 0 20 035 & & Amount and type of ice b\textsubscript{i} & Code table, 0\tabularnewline
& 0 20 036 & & Ice situation z\textsubscript{i} & Code table, 0\tabularnewline
& 0 20 037 & & Ice development S\textsubscript{i} & Code table, 0\tabularnewline
& 0 20 038 & & Bearing of ice edge D\textsubscript{i} & Degree true, 0\tabularnewline
& & & \textbf{Ship marine data} &\tabularnewline
\textbf{3 02 057} & \textbf{3 02 056} & & \textbf{Sea/water temperature (method of measurement, and depth below sea surface)} &\tabularnewline
& & 0 02 038 & Method of water temperature measurement & Code table, 0\tabularnewline
\begin{minipage}[t]{0.17\columnwidth}\raggedright
\strut
\end{minipage} & \begin{minipage}[t]{0.17\columnwidth}\raggedright
\strut
\end{minipage} & \begin{minipage}[t]{0.17\columnwidth}\raggedright
0 07 063\strut
\end{minipage} & \begin{minipage}[t]{0.17\columnwidth}\raggedright
Depth below sea/water surface (cm)

(for sea-surface temperature measurement)\strut
\end{minipage} & \begin{minipage}[t]{0.17\columnwidth}\raggedright
m, 2\strut
\end{minipage}\tabularnewline
& & 0 22 043 & Sea/water temperature s\textsubscript{s}T\textsubscript{w}T\textsubscript{w}T\textsubscript{w} & K, 2\tabularnewline
& & 0 07 063 & Depth below sea/water surface (cm) (set to missing to cancel the previous value) & m, 2\tabularnewline
& & & \emph{Waves} &\tabularnewline
& 3 02 021 & 0 22 001 & Direction of waves & Degree true, 0\tabularnewline
& & 0 22 011 & Period of waves P\textsubscript{wa}P\textsubscript{wa} & s, 0\tabularnewline
& & 0 22 021 & Height of waves H\textsubscript{wa}H\textsubscript{wa} & m, 1\tabularnewline
& 3 02 024 & 0 22 002 & Direction of wind waves & Degree true, 0\tabularnewline
& & 0 22 012 & Period of wind waves P\textsubscript{w}P\textsubscript{w} & s, 0\tabularnewline
& & 0 22 022 & Height of wind waves H\textsubscript{w}H\textsubscript{w} & m, 1\tabularnewline
& & 1 01 002 & Replicate 1 descriptor 2 times &\tabularnewline
\begin{minipage}[t]{0.17\columnwidth}\raggedright
\strut
\end{minipage} & \begin{minipage}[t]{0.17\columnwidth}\raggedright
\strut
\end{minipage} & \begin{minipage}[t]{0.17\columnwidth}\raggedright
3 02 023\strut
\end{minipage} & \begin{minipage}[t]{0.17\columnwidth}\raggedright
Swell waves (2 systems of swell)

d\textsubscript{w1}d\textsubscript{w1}, P\textsubscript{w1}P\textsubscript{w1}, H\textsubscript{w1}H\textsubscript{w1}

d\textsubscript{w2}d\textsubscript{w2}, P\textsubscript{w2}P\textsubscript{w2}, H\textsubscript{w2}H\textsubscript{w2}\strut
\end{minipage} & \begin{minipage}[t]{0.17\columnwidth}\raggedright
\strut
\end{minipage}\tabularnewline
& & & \textbf{Ship ``period'' data} &\tabularnewline
& & & \emph{Present and past weather} &\tabularnewline
\textbf{3 02 060} & \textbf{3 02 038} & 0 20 003 & Present weather ww & Code table, 0\tabularnewline
& & 0 04 024 & Time period or displacement (in hours) & Hour, 0\tabularnewline
& & 0 20 004 & Past weather (1) W\textsubscript{1} & Code table, 0\tabularnewline
& & 0 20 005 & Past weather (2) W\textsubscript{2} & Code table, 0\tabularnewline
& & & \emph{Precipitation measurement} &\tabularnewline
& \textbf{3 02 040} & 0 07 032 & \vtop{\hbox{\strut Height of sensor above local ground (or deck of marine platform)}\hbox{\strut (for precipitation measurement)}} & m, 2\tabularnewline
& & 1 02 002 & Replicate 2 descriptors 2 times &\tabularnewline
& & 0 04 024 & Time period or displacement (in hours) t\textsubscript{R} & Hour, 0\tabularnewline
& & 0 13 011 & Total precipitation/total water equivalent (of snow) RRR & kg m\textsuperscript{--2}, 1\tabularnewline
& & & \emph{Ship extreme temperature data} &\tabularnewline
& \textbf{3 02 058} & 0 07 032 & \vtop{\hbox{\strut Height of sensor above local ground (or deck of marine platform)}\hbox{\strut (for temperature measurement)}} & m, 2\tabularnewline
\begin{minipage}[t]{0.17\columnwidth}\raggedright
\strut
\end{minipage} & \begin{minipage}[t]{0.17\columnwidth}\raggedright
\strut
\end{minipage} & \begin{minipage}[t]{0.17\columnwidth}\raggedright
0 07 033\strut
\end{minipage} & \begin{minipage}[t]{0.17\columnwidth}\raggedright
Height of sensor above water surface

(for temperature measurement)\strut
\end{minipage} & \begin{minipage}[t]{0.17\columnwidth}\raggedright
m, 1\strut
\end{minipage}\tabularnewline
& & 0 04 024 & Time period or displacement & Hour, 0\tabularnewline
& & 0 04 024 & Time period or displacement (see Notes 3 and 4) & Hour, 0\tabularnewline
& & 0 12 111 & Maximum temperature, at height and over period specified s\textsubscript{n}T\textsubscript{x}T\textsubscript{x}T\textsubscript{x} & K, 2\tabularnewline
& & 0 04 024 & Time period or displacement & Hour, 0\tabularnewline
& & 0 04 024 & Time period or displacement (see Note 4) & Hour, 0\tabularnewline
& & 0 12 112 & Minimum temperature, at height and over period specified s\textsubscript{n}T\textsubscript{n}T\textsubscript{n}T\textsubscript{n} & K, 2\tabularnewline
& & & \emph{Ship wind data} &\tabularnewline
& \textbf{3 02 059} & 0 07 032 & \vtop{\hbox{\strut Height of sensor above local ground (or deck of marine platform)}\hbox{\strut (for wind measurement)}} & m, 2\tabularnewline
& & 0 07 033 & \vtop{\hbox{\strut Height of sensor above water surface}\hbox{\strut (for wind measurement)}} & m, 1\tabularnewline
& & 0 02 002 & Type of instrumentation for wind measurement i\textsubscript{w} & Flag table, 0\tabularnewline
& & 0 08 021 & Time significance (= 2 Time averaged) & Code table, 0\tabularnewline
& & 0 04 025 & Time period or displacement (= --10 minutes, or number of minutes after a significant change of wind) & Minute, 0\tabularnewline
& & 0 11 001 & Wind direction dd & Degree true, 0\tabularnewline
& & 0 11 002 & Wind speed ff & m s\textsuperscript{--1}, 1\tabularnewline
& & 0 08 021 & Time significance (= missing value) & Code table, 0\tabularnewline
& & 1 03 002 & Replicate 3 descriptors 2 times &\tabularnewline
& & 0 04 025 & Time period or displacement (in minutes) & Minute, 0\tabularnewline
& & 0 11 043 & Maximum wind gust direction & Degree true, 0\tabularnewline
& & 0 11 041 & Maximum wind gust speed 910f\textsubscript{m}f\textsubscript{m}, 911f\textsubscript{x}f\textsubscript{x} & m s\textsuperscript{--1}, 1\tabularnewline
\bottomrule
\end{longtable}

Notes:

(1) 0~01~012: Means course made good (average course over the ground) during the three hours preceding the time of observation.

(2) 0~01~013: Means speed made good (average speed over the ground) during the three hours preceding the time of observation.

(3) Within RA IV, the maximum temperature at 1200 UTC is reported for the previous calendar day (i.e. the ending time of the period is not equal to the nominal time of the report). To construct the required time range, descriptor 0~04~024 has to be included two times. If the period ends at the nominal time of the report, value of the second 0~04~024 shall be set to 0.

(4) Within RA III, the maximum daytime temperature and the minimum night-time temperature is reported (i.e. the ending time of the period may not be equal to the nominal time of the report). To construct the required time range, descriptor 0~04~024 has to be included two times. If the period ends at the nominal time of the report, value of the second 0~04~024 shall be set to 0.

(5) If ``plain language'' text is reported within Section 2, this information can be conveyed in BUFR via the use of an appropriate 2~05~YYY field as an extra descriptor following the above basic template.

(6) If WMO block and station number is to be included in reports from a fixed sea station, sequence descriptor \textless3~08~009\textgreater{} may be preceded by sequence descriptor \textless3~01~001\textgreater.

\textbf{\\
Regulations:}

\textbf{B/C10.1 Section 1 of BUFR or CREX}

\textbf{B/C10.2 Ship identification, movement, date/time, horizontal and vertical coordinates}

\textbf{B/C10.3 Pressure and 3-hour pressure change}

\textbf{B/C10.4 Ship ``instantaneous'' data}

\textbf{B/C10.4.1 Ship temperature and humidity data}

\textbf{B/C10.4.2 Ship visibility data}

\textbf{B/C10.4.3 Precipitation past 24 hours}

B/C10.4.4 General cloud information

\textbf{B/C10.4.5 Individual cloud layers or masses}

\textbf{B/C10.5 Icing and ice}

\textbf{B/C10.6 Ship marine data}

B/C10.7 ``Instantaneous'' data required by regional or national reporting practices

\textbf{B/C10.8 Ship ``period'' data}

\textbf{B/C10.8.1 Present and past weather}

B/C10.8.2 Precipitation measurement

\textbf{B/C10.8.3 Ship extreme temperature data}

\textbf{B/C10.8.4 Ship wind data}

\textbf{B/C10.9 ``Period'' data required by regional or national reporting practices}

\textbf{B/C10.1 Section 1 of BUFR or CREX}

\textbf{B/C10.1.1 Entries required in Section 1 of BUFR}

\begin{quote}
\textbf{The following entries shall be included in BUFR Section 1:}

-- \textbf{BUFR master table;}

-- \textbf{Identification of originating/generating centre;}

-- \textbf{Identification of originating/generating sub-centre;}

-- \textbf{Update sequence number;}

-- \textbf{Identification of inclusion of optional section;}

-- \textbf{Data category (= 001 for SHIP data);}

-- \textbf{International data sub-category (see Notes 1 and 2);}

-- \textbf{Local data sub-category;}

-- \textbf{Version number of master table;}

-- \textbf{Version number of local tables;}

-- \textbf{Year (year of the century up to BUFR edition 3);}

-- \textbf{Month (standard time);}

-- \textbf{Day (standard time = YY in the} abbreviated telecommunication header \textbf{for SHIP data});

-- \textbf{Hour (standard time = GG in the} abbreviated telecommunication header \textbf{for SHIP data});

-- \textbf{Minute (standard time = 00 for SHIP data);}

-- \textbf{Second (= 0) (see Note 1).}

\textbf{Notes:}

\textbf{(1) Inclusion of this entry is required starting with BUFR edition 4.}

\textbf{(2) If required, the international data sub-category shall be included for SHIP data as 000 at all observation times 00, 01, 02, ..., 23 UTC.}

\textbf{(3) If an NMHS performs conversion of SHIP data produced by another NMHS, o}riginating centre in Section 1 shall indicate \textbf{the converting centre and} originating sub-centre shall indicate the \textbf{producer of SHIP bulletins. Producer of SHIP bulletins shall be specified in Common Code table C-12 as a sub-centre of the originating centre, i.e. of the NMHS executing the conversion.}
\end{quote}

\textbf{B/C10.1.2 Entries required in Section 1 of CREX}

\begin{quote}
\textbf{The following entries shall be included in CREX Section 1:}

-- \textbf{CREX master table;}

-- \textbf{CREX edition number;}

-- \textbf{CREX table version number;}

-- \textbf{Version number of BUFR master table (see Note 1);}

-- \textbf{Version number of local tables (see Note 1);}

-- \textbf{Data category (= 001 for SHIP data);}

-- \textbf{International data sub-category (see Notes 1 and 2);}

-- \textbf{Identification of originating/generating centre (see Note 1);}

-- \textbf{Identification of originating/generating sub-centre (see Note 1);}

-- \textbf{Update sequence number (see Note 1);}

-- \textbf{Number of subsets (see Note 1);}

-- \textbf{Year (standard time) (see Note 1);}

-- \textbf{Month (standard time) (see Note 1);}

-- \textbf{Day (standard time = YY in the} abbreviated telecommunication header \textbf{for SHIP data}) \textbf{(see Note 1);}

-- \textbf{Hour (standard time = GG in the} abbreviated telecommunication header \textbf{for SHIP data}) \textbf{(see Note 1);}

-- \textbf{Minute (standard time = 00 for SHIP data) (see Note 1).}

\textbf{Notes:}

\textbf{(1) Inclusion of these entries is required starting with CREX edition 2.}

\textbf{(2) If inclusion of international data sub-category is required, Note 2 under B/C10.1.1 applies.}

\textbf{(3) If an NMHS performs conversion of SHIP data produced by another NMHS, Note 3 under B/C10.1.1 applies.}
\end{quote}

\textbf{B/C10.2 Ship identification, movement, date/time, horizontal and vertical coordinates \textless3~01~093\textgreater{}}

\textbf{B/C10.2.1 Ship identification, movement, type of station}

\begin{quote}
Ship identifier (0~01~011) shall be always reported as a non-missing value. In the absence of a suitable call sign, the word SHIP shall be used for ship identifier in reports of sea stations other than buoys, drilling rigs and oil- and gas-production platforms. {[}12.1.7(b){]}

\emph{If required, WMO block number (0}~\emph{01~001) and WMO station number (0}~\emph{01~002) may be included in reports from a fixed sea station.}

Note: Note 6 under TM~308009 shall apply.
\end{quote}

\textbf{B/C10.2.2 Ship movement}

\begin{quote}
Direction of motion of moving observing platform (0~01~012) shall be reported in degrees true to indicate course made good (average course over the ground) during the three hours preceding the time of observation.

Speed of motion of moving observing platform (0~01~013) shall be reported in metres per second to indicate speed made good (average speed over the ground) during the three hours preceding the time of observation.
\end{quote}

\textbf{B/C10.2.2.1} Direction and speed of motion of moving observing platform shall always be included in reports from stations, which have observed maritime conditions, and in reports from ships being requested to include this information as a routine procedure. {[}12.3.1.1{]}

\textbf{B/C10.2.2.2} Direction and speed of motion of moving observing platform may be included as missing values in reports from ships that have not been directly recruited and instrumented by an NMHS, except when reporting from an area for which the ship report collecting centre, in order to meet a requirement of a search and rescue centre, has requested inclusion of direction and speed of ship motion as a routine procedure. {[}12.3.1.2(b){]}

\textbf{B/C10.2.2.3} Stationary position of ship shall be reported by 0~01~012 set to 0 and 0~01~013 set to 0. Course of ship unknown (D\textsubscript{s} = 9) shall be reported by 0~01~012 set to 509.

\textbf{B/C10.2.3 Type of station}

\begin{quote}
Type of station (0~02~001) shall be reported to indicate the type of the station operation (manned, automatic or hybrid).

Note: If a station operates as a manned station for a part of the day and as an automatic station for the rest of the day, code figure 2 (Hybrid) may be used in all reports. It is preferable, however, to use code figure 1 (manned) in reports produced under the supervision of an observer, and a code figure 0 (Automatic) in reports produced while the station operates in the automatic mode.
\end{quote}

\textbf{B/C10.2.4 Time of observation}

\begin{quote}
Year (0~04~001), month (0~04~002), day (0~04~003), hour (0~04~004) and minute (0~04~005) of the actual time of observation shall be reported.

Note: The actual time of observation shall be the time at which the barometer is read. {[}12.1.6{]}
\end{quote}

\textbf{B/C10.2.4.1} If the actual time of observation differs by 10 minutes or less from the standard time reported in Section 1, the standard time may be reported instead of the actual time of observation. {[}12.2.8{]}

\textbf{B/C10.2.5 Horizontal and vertical coordinates}

\begin{quote}
\textbf{Latitude (0~05~002) and longitude} (0\textbf{~}06~002) of the station shall be reported in degrees with precision in hundredths of a degree.

Height of station ground above mean sea level (0\textbf{~}07~030) and height of barometer above mean sea level (0\textbf{~}07~031) shall be reported in metres with precision in tenths of a metre.
\end{quote}

\textbf{B/C10.3 Pressure and 3-hour pressure change \textless3~02~001\textgreater{}}

\textbf{B/C10.3.1 Pressure at the station level}

\begin{quote}
Pressure at the station level (0~10~004), i.e. at the level defined by 0~07~031 (height of barometer above mean sea level), shall be reported in pascals (with precision in tens of pascals).

Note: Inclusion of the station pressure in reports from sea stations is left to the decision of individual Members.
\end{quote}

\textbf{B/C10.3.2 Pressure} \textbf{reduced to mean sea level}

\begin{quote}
Pressure reduced to mean sea level (0~10~051) shall be reported in pascals (with precision in tens of pascals).
\end{quote}

\textbf{B/C10.3.2.1} In reports from ships, air pressure at mean sea level shall be reported. {[}12.1.3.6{]}, {[}12.1.3.7{]}

\textbf{B/C10.3.3 Three-hour pressure change and characteristic of pressure tendency}

\begin{quote}
Amount of pressure change at station level, during the three hours preceding the time of observation (0~10~061), either positive, zero \emph{or negative}, shall be reported in pascals (with precision in tens of pascals).
\end{quote}

\textbf{B/C10.3.3.1} Unless specified otherwise by regional decision, pressure tendency shall be included whenever the three-hourly pressure tendency is available. {[}12.2.3.5.1{]}

\textbf{B/C10.3.3.2} The characteristic of pressure tendency (Code table 0~10~063) over the past three hours shall, whenever possible, be determined on the basis of pressure samples at equi-spaced intervals not exceeding one hour.

\begin{quote}
Note: Algorithms for selecting the appropriate code figure are included in the \emph{Guide to Meteorological Instruments and Methods of Observation} (WMO-No. 8).

{[}12.2.3.5.2{]}
\end{quote}

\textbf{B/C10.3.3.3} Where it is not possible to apply the algorithms specified in Regulation B/C10.3.3.2 in reports from automatic weather stations, the characteristic of pressure tendency shall be reported as 2 when the tendency is positive, as 7 when the tendency is negative, and as 4 when the atmospheric pressure is the same as three hours before. {[}12.2.3.5.3{]}

\textbf{B/C10.4 Ship ``instantaneous'' data \textless3~02~054\textgreater{}}

\textbf{B/C10.4.1 Ship temperature and humidity data \textless3~02~052\textgreater{}}

\textbf{B/C10.4.1.1 Height of sensor above marine deck platform and height of sensor above water surface}

\begin{quote}
Height of sensor above marine deck platform (0~07~032) for temperature and humidity measurement shall be reported in metres (with precision in hundredths of a metre).

This datum represents the actual height of temperature and humidity sensors above marine deck platform at the point where the sensors are located.

Height of sensor above water surface (0~07~033) for temperature and humidity measurement shall be reported in metres (with precision in tenths of a metre).

This datum represents the actual height of temperature and humidity sensors above marine water surface of sea or lake.
\end{quote}

\textbf{B/C10.4.1.2 Dry-bulb air temperature}

\begin{quote}
Dry-bulb air temperature (0~12~101) shall be reported in kelvin (with precision in hundredths of a kelvin); if produced in CREX, in degrees Celsius (with precision in hundredths of a degree Celsius).

Notes:

(1) Temperature data shall be reported with precision in hundredths of a degree even if they are measured with the accuracy in tenths of a degree. This requirement is based on the fact that conversion from the Kelvin to the Celsius scale has often resulted into distortion of the data values.

(2) Temperature t (in degrees Celsius) shall be converted into temperature T (in kelvin) using equation: T = t + 273.15.
\end{quote}

\textbf{B/C10.4.1.2.1} When the data are not available as a result of a temporary instrument failure, this quality shall be included as a missing value. {[}12.2.3.2{]}

\textbf{B/C10.4.1.3 Wet-bulb temperature and method of its measurement}

\begin{quote}
Wet-bulb temperature (0~12~102) shall be reported in kelvin (with precision in hundredths of a kelvin); if produced in CREX, in degrees Celsius (with precision in hundredths of a degree Celsius). Method of wet-bulb temperature measurement shall be reported by the preceding entry (Code table 0~02~039). Wet-bulb temperature data shall be reported with precision in hundredths of a degree even if they are available with the accuracy in tenths of a degree.

Note: Notes 1 and 2 under Regulation B/C10.4.1.2 shall apply.
\end{quote}

\textbf{B/C10.4.1.3.1} When wet-bulb temperature is used to derive dewpoint value in a ship report, 0~12~102 shall be included to report the wet-bulb temperature measurement. {[}12.3.6{]}

\textbf{B/C10.4.1.4 Dewpoint temperature}

\begin{quote}
When available, dewpoint temperature (0~12~103) shall be reported in kelvin (with precision in hundredths of a kelvin); if produced in CREX, in degrees Celsius (with precision in hundredths of a degree Celsius).

Note: Notes 1 and 2 under Regulation B/C10.4.1.2 shall apply.
\end{quote}

\textbf{B/C10.4.1.5 Relative humidity}

\begin{quote}
Relative humidity (0~13~003) shall be reported in units of a per cent.
\end{quote}

\textbf{B/C10.4.1.5.1} \emph{Both dewpoint temperature and relative humidity shall be reported when available.}

\textbf{B/C10.4.2 Ship visibility data \textless3~02~053\textgreater{}}

\textbf{B/C10.4.2.1 Height of sensor above marine deck platform and height of sensor above water surface}

\begin{quote}
Height of sensor above marine deck platform (0~07~032) for visibility measurement shall be reported in metres (with precision in hundredths of a metre).

This datum represents the actual height of visibility sensors above marine deck platform at the point where the sensors are located. If visibility is estimated by a human observer, the average height of observer's eyes above marine deck platform shall be reported.

Height of sensor above water surface (0~07~033) for visibility measurement shall be reported in metres (with precision in tenths of a metre).

This datum represents the actual height of visibility sensors above the level of water surface of sea or lake. If visibility is estimated by a human observer, the average height of observer's eyes above the level of water surface of sea or lake at the time of observation shall be reported.
\end{quote}

\textbf{B/C10.4.2.2 Horizontal visibility}

\begin{quote}
Horizontal visibility (0~20~001) at surface shall be reported in metres (with precision in tens of metres).
\end{quote}

\textbf{B/C10.4.2.2.1} When the horizontal visibility is not the same in different directions, the shortest distance shall be given for visibility. {[}12.2.1.3.1{]}

\textbf{B/C10.4.2.2.2} Horizontal visibility greater than 81 900 m shall be expressed by 0~20~001 set to 81~900 m; if TDCF data are converted from SHIP data, 0~20~001 set to 50~000~m shall indicate horizontal visibility equal to or greater than 50~000 m. {[}12.2.1.3.2{]}

\textbf{B/C10.4.3 Precipitation past 24 hours \textless3~02~034\textgreater{}}

\textbf{B/C10.4.3.1 Height of sensor above marine deck platform}

\begin{quote}
Height of sensor above marine deck platform (0~07~032) for precipitation measurement shall be reported in metres (with precision in hundredths of a metre).

This datum represents the actual height of the rain gauge rim above marine deck platform at the point where the rain gauge is located.

Note: Height of sensor above water surface (0~07~033) is not required for precipitation measurement. Therefore, there is an entry 0~07~033, directly preceding the sequence 3~02~034, that is set to a missing value to cancel the previous value.
\end{quote}

\textbf{B/C10.4.3.2 Total amount of precipitation during the 24-hour period}

\begin{quote}
Total amount of precipitation during the 24-hour period ending at the time of observation (0~13~023) shall be reported in kilograms per square metre (with precision in tenths of a kilogram per square metre). {[}12.4.9{]}
\end{quote}

\textbf{B/C10.4.3.2.1} The precipitation over the past 24 hours shall be included (not missing) at least once a day at one appropriate time of the main standard times (0000, 0600, 1200, 1800 UTC). {[}12.4.1{]}

\textbf{B/C10.4.3.2.2} Precipitation, when it can be and has to be reported, shall be reported as 0.0~kg~m\textsuperscript{--2} if no precipitation were observed during the referenced period. {[}12.2.5.4{]}

\textbf{B/C10.4.3.2.3} Trace shall be reported as ``--0.1 kg~m\textsuperscript{--2}''.

\textbf{B/C10.4.4 General cloud information \textless3~02~004\textgreater{}}

\textbf{B/C10.4.4.1 Total cloud cover}

\begin{quote}
\emph{Total cloud cover (0~20~010) shall embrace the total fraction of the celestial dome covered by clouds irrespective of their genus. It shall be reported} in \emph{units of a per cent}.

Notes:

(1) Total cloud cover shall be reported as 113 when sky is obscured by fog and/or other meteorological phenomena.

(2) When cloud cover is observed in oktas the cloud cover shall be converted to per cent, with fractional numbers rounded up (e.g. 1 okta = 12.5\%, rounded to 13 \%).
\end{quote}

\textbf{B/C10.4.4.1.1} Total cloud cover shall be reported as actually seen by the observer during the observation. {[}12.2.2.2.1{]}

\textbf{B/C10.4.4.1.2} Altocumulus perlucidus or Stratocumulus perlucidus (``mackerel sky'') shall be reported \emph{as 99\% or less} (unless overlying clouds appear to cover the whole sky) since breaks are always present in this cloud form even if it extends over the whole celestial dome. {[}12.2.2.2.2{]}

\textbf{B/C10.4.4.1.3} Total cloud cover shall be reported as zero when blue sky or stars are seen through existing fog or other analogous phenomena without any trace of cloud being seen. {[}12.2.2.2.3{]}

\textbf{B/C10.4.4.1.4} When clouds are observed through fog or analogous phenomena, their amount shall be evaluated and reported as if these phenomena were non-existent. {[}12.2.2.2.4{]}

\textbf{B/C10.4.4.1.5} Total cloud cover shall not include the amount resulting from rapidly dissipating condensation trails. {[}12.2.2.2.5{]}

\textbf{B/C10.4.4.1.6} Persistent condensation trails and cloud masses, which have obviously developed from condensation trails, shall be reported as cloud. {[}12.2.2.2.6{]}

\textbf{B/C10.4.4.2 Vertical significance} -- Code table 0~08~002

\begin{quote}
To specify vertical significance (0~08~002) within the sequence 3~02~004, a code figure shall be selected in the following way:

(a) If low clouds are observed, then code figure 7 (Low cloud) shall be used;

(b) If there are no low clouds but middle clouds are observed, then code figure 8 (Middle clouds) shall be used;

(c) If there are no low and there are no middle clouds but high clouds are observed, then code figure 0 shall be used;

(d) If sky is obscured by fog and/or other phenomena, then code figure 5 (Ceiling) shall be used;

(e) If there are no clouds (clear sky), then code figure 62 (Value not applicable) shall be used

(f) If the cloud cover is not discernible for reasons other than (d) above or observation is not made, then code figure 63 (Missing value) shall be used.
\end{quote}

\textbf{B/C10.4.4.3 Cloud amount (of low or middle clouds}) -- Code table 0~20~011

\begin{quote}
\emph{Amount of all the low clouds (clouds of the genera Stratocumulus, Stratus, Cumulus, and Cumulonimbus) present or, if no low clouds are present, the amount of all the middle clouds (clouds of the genera Altocumulus, Altostratus, and Nimbostratus) present}.
\end{quote}

\textbf{B/C10.4.4.3.1} Cloud amount shall be reported as follows:

\begin{quote}
(a) If there are low clouds, then the total amount of all low clouds, as actually seen by the observer during the observation shall be reported for the cloud amount;

(b) If there are no low clouds but there are middle clouds, then the total amount of the middle clouds shall be reported for the cloud amount;

(c) If there are no low clouds and there are no middle clouds but there are high clouds (clouds of the genera Cirrus, Cirrocumulus, and Cirrostratus), then the cloud amount shall be reported as zero;

{[}12.2.7.2.1{]}

\textbf{(d) If no clouds are observed (clear sky), then the cloud amount shall be reported as 0;}

\textbf{(e) If sky is obscured by fog and/or other meteorological phenomena, then the cloud amount shall be reported as 9;}

\textbf{(f) If cloud cover is indiscernible for reasons other than fog or other meteorological phenomena, or observation is not made, the cloud amount shall be reported as missing.}
\end{quote}

\textbf{B/C10.4.4.3.2} Amount of Altocumulus perlucidus or Stratocumulus perlucidus (``mackerel sky'') shall be reported using code figure 7 or less since breaks are always present in this cloud form even if it extends over the whole celestial dome. {[}12.2.7.2.2{]}

\textbf{B/C10.4.4.3.3} When the clouds reported for cloud amount are observed through fog or an analogous phenomenon, the cloud amount shall be reported as if these phenomena were not present. {[}12.2.7.2.3{]}

\textbf{B/C10.4.4.3.4} If the clouds reported for cloud amount include contrails, then the cloud amount shall include the amount of persistent contrails. Rapidly dissipating contrails shall not be included in the value for the cloud amount. {[}12.2.7.2.4{]}

\textbf{B/C10.4.4.4 Height of base of lowest cloud}

\begin{quote}
\emph{Height above surface of the base (0~20~013) of the lowest cloud seen shall be reported} in metres (with precision in tens of metres).

Note: The term~«~height above surface~»~shall be considered as being the height above water surface of sea or lake.
\end{quote}

\textbf{B/C10.4.4.4.1} When clouds are observed through fog or analogous phenomena but the sky is discernible, the base of the lowest cloud shall refer to the base of the lowest cloud observed, if any. When, under the above conditions, the sky is not discernible, the base of the lowest cloud shall be replaced by vertical visibility. {[}12.4.10.5{]}

\textbf{B/C10.4.4.4.2} \emph{When no cloud is reported (total cloud cover = 0)} the base of the lowest cloud \emph{shall be reported as a missing value.}

\textbf{B/C10.4.4.4.3} \emph{If synoptic data are produced in BUFR or CREX by conversion from a TAC report, the following approach shall be used: Height of base of the lowest cloud 0}~\emph{20}~\emph{013 shall be derived from the h\textsubscript{s}h\textsubscript{s} in the first group 8 in section 3, i.e. from the h\textsubscript{s}h\textsubscript{s} of the lowest cloud. If and only if groups 8 are not reported in section 3, 0}~\emph{20}~\emph{013 may be derived from h. The lower limit of the range defined for h\textsubscript{s}h\textsubscript{s} and for h shall be used. However, if groups 8 are not reported in section 3 and h = 9 and N\textsubscript{h}} ≠ 0, then 0 \emph{20}~\emph{013 shall be 4}~\emph{000 m; if groups 8 are not reported in section 3 and h = 9 and N\textsubscript{h} = 0, then 0}~\emph{20}~\emph{013 shall be 8}~\emph{000 m.}

\textbf{B/C10.4.4.5 Cloud type of low, middle and high clouds} -- Code table 0~20~012

\begin{quote}
Clouds of the genera Stratocumulus, Stratus, Cumulus, and Cumulonimbus (low clouds) shall be reported for the first entry 0~20~012, clouds of the genera Altocumulus, Altostratus, and Nimbostratus (middle clouds) shall be reported for the second entry 0~20~012 and clouds of the genera Cirrus, Cirrocumulus, and Cirrostratus (high clouds) shall be reported for the third entry 0~20~012.
\end{quote}

\textbf{B/C10.4.4.5.1} The reporting of type of low, middle and high clouds shall be as specified in the \emph{International Cloud Atlas} (WMO-No. 407), Volume I. {[}12.2.7.3{]}

\textbf{B/C10.4.5 Individual cloud layers or masses}

\textbf{B/C10.4.5.1 Number of individual cloud layers or masses}

\begin{quote}
The number of individual cloud layers or masses shall be indicated by Delayed descriptor replication factor 0~31~001 in BUFR and by a four-digit number in the Data Section corresponding to the position of the replication descriptor in the Data Description Section of CREX.

Notes:

(1) The number of cloud layers or masses shall never be set to missing value.

(2) The number of cloud layers or masses shall be set to a positive value in a NIL report.

(3) If data compression is to be used, BUFR Regulation 94.6.3, Note 2, sub-note ix shall apply.
\end{quote}

\textbf{B/C10.4.5.1.1} When reported from a manned station, the number of individual cloud layers or masses shall in the absence of Cumulonimbus clouds not exceed three. Cumulonimbus clouds, when observed, shall always be reported, so that the total number of individual cloud layers or masses can be four. The selection of layers (or masses) to be reported shall be made in accordance with the following criteria:

\begin{quote}
(a) The lowest individual layer (or mass) of any amount (cloud amount at least one octa or less, but not zero);

(b) The next higher individual layer (or mass) the amount of which is greater than two octas;

(c) The next higher individual layer (or mass) the amount of which is greater than four octas;

(d) Cumulonimbus clouds, whenever observed and not reported under (a), (b) and (c) above.

{[}12.4.10.1{]}
\end{quote}

\textbf{B/C10.4.5.1.2} When the sky is clear, the number of individual cloud layers or masses shall be set to zero.

\textbf{B/C10.4.5.1.3} The order of reporting the individual cloud layers or masses shall always be from lower to higher levels. {[}12.4.10.2{]}

\textbf{B/C10.4.5.2 Individual cloud layer or mass \textless3~02~005\textgreater{}}

\begin{quote}
Each cloud layer or mass shall be represented by the following four parameters: Vertical significance (0~08~002), amount of individual cloud layer or mass (0~20~011), type of cloud layer or mass (0~20~012) and height of base of individual cloud layer or mass (0~20~013).
\end{quote}

\textbf{B/C10.4.5.2.1 Vertical significance} -- Code table 0~08~002

\begin{quote}
To specify vertical significance (0~08~002) within the sequence 3~02~005, a code figure shall be selected in the following way:

(a) Code figure 1 shall be used in the first non-Cumulonimbus layer;

(b) Code figure 2 shall be used in the second non-Cumulonimbus layer;

(c) Code figure 3 shall be used in the third non-Cumulonimbus layer;

(d) Code figure 4 shall be used in any Cumulonimbus layer;

(e) If sky is obscured by fog and/or other phenomena, then code figure 5 (Ceiling) shall be used;

(f) If the cloud cover is not discernible for reasons other than (e) above or observation is not made, then code figure 63 (Missing value) shall be used;

(g) If a station operates in the automatic mode and is sufficiently equipped, code figure 21, 22, 23 and 24 shall be used to identify the first, the second, the third and the fourth instrument detected cloud layer, respectively;

(h) If a station operates in the automatic mode and no clouds are detected by the cloud detection system, code figure 20 shall be used.
\end{quote}

\textbf{B/C10.4.5.2.2 Cloud amount, type and height of base}

\textbf{B/C10.4.5.2.2.1} When the sky is clear, in accordance with Regulation B/C10.4.5.1.2 cloud amount, genus, and height shall not be included. {[}12.4.10.4{]}

\textbf{B/C10.4.5.2.2.2} In determining cloud amounts (Code table 0~20~011) to be reported for individual layers or masses, the observer shall estimate, by taking into consideration the evolution of the sky, the cloud amounts of each individual layer or mass at the different levels, as if no other clouds existed. {[}12.4.10.3{]}

\textbf{B/C10.4.5.2.2.3} Type of a cloud layer or mass (Code table 0~20~012) shall be reported using code figures 0, 1, 2, 3, 4, 5, 6, 7, 8, 9, 59 and 63.

\textbf{B/C10.4.5.2.2.4} If, notwithstanding the existence of fog or other obscuring phenomena, the sky is discernible, the partially obscuring phenomena shall be disregarded. If, under the above conditions, the sky is not discernible, the cloud type shall be reported using \emph{code figure 59} and the cloud height shall be replaced by vertical visibility.

\begin{quote}
Note: The vertical visibility is defined as the vertical visual range into an obscuring medium.

{[}12.4.10.5{]}
\end{quote}

\textbf{B/C10.4.5.2.2.5} If two or more types of cloud occur with their bases at the same level and this level is one to be reported in accordance with Regulation B/C10.4.5.1.1, the selection for cloud type and amount shall be made with the following criteria:

\begin{quote}
(a) If these types do not include Cumulonimbus then cloud genus shall refer to the cloud type that represents the greatest amount, or if there are two or more types of cloud all having the same amount, the highest applicable code figure for cloud genus shall be reported. Cloud amount shall refer to the total amount of cloud whose bases are all at the same level;

(b) If these types do include Cumulonimbus then one layer shall be reported to describe only this type with cloud genus indicated as Cumulonimbus and the cloud amount as the amount of the Cumulonimbus. If the total amount of the remaining type(s) of cloud (excluding Cumulonimbus) whose bases are all at the same level is greater than that required by Regulation B/C10.4.5.1.1, then another layer shall be reported with type being selected in accordance with (a) and amount referring to the total amount of the remaining cloud (excluding Cumulonimbus).

{[}12.4.10.6{]}
\end{quote}

\textbf{B/C10.4.5.2.2.6} Regulations B/C10.4.4.1.3 to B/C10.4.4.1.6, inclusive, shall apply. {[}12.4.10.7{]}

\textbf{B/C10.4.5.2.2.7} \emph{Height above surface of the cloud base (0~20~013) shall be reported} in metres (with precision in tens of metres).

\begin{quote}
Note: The term~«~height above surface~»~shall be considered as being the height above water surface of sea or lake.
\end{quote}

\textbf{B/C10.5 Icing and ice \textless3~02~055\textgreater{}}

\textbf{B/C10.5.1 Icing}

\begin{quote}
Thickness of ice deposit (0~20~031) shall be reported in metres (with precision in hundredths of a metre).

Rate of ice accretion (0~20~032) shall be reported using corresponding Code table.

Cause of ice accretion (0~20~033) shall be reported using corresponding Flag table.
\end{quote}

\textbf{B/C10.5.1.1} When the ice accretion on ships is reported in plain language, this information shall be conveyed in BUFR/CREX via the use of an appropriate 2~05~YYY field as an extra descriptor following the basic template.

\textbf{B/C10.5.1.2} When the ice accretion on ships is reported in plain language, it shall be preceded by the word ICING. {[}12.3.5{]}

\textbf{\\
}

\textbf{B/C10.5.2 Ice}

\begin{quote}
Sea-ice concentration (0~20~034) shall be reported using corresponding Code table.

Amount and type of ice (0~20~035) shall be reported using corresponding Code table.

Ice situation (0~20~036) shall be reported using corresponding Code table.

Ice development (0~20~037) shall be reported using corresponding Code table.

Bearing of ice edge (0~20~038) shall be reported in degrees true.
\end{quote}

\textbf{B/C10.5.2.1} The reporting of sea ice and ice of land origin using the sequence \textless0~20~034, 0~20~035, 0~20~036, 0~20~037, 0~20~038\textgreater{} shall not supersede the reporting of sea ice and icebergs in accordance with the International Convention for the Safety of Life at Sea. {[}12.3.7.1{]}

\textbf{B/C10.5.2.2} The sequence \textless0~20~034, 0~20~035, 0~20~036, 0~20~037, 0~20~038\textgreater{} shall be reported whenever sea ice and/or ice of land origin are observed from the ship's position at the time of observation, unless the ship is required to report ice conditions by means of a special sea-ice code. {[}12.3.7.2{]}

\textbf{B/C10.5.2.3} When an ice edge is crossed or sighted between observational hours, it shall be reported as a plain-language addition in the form ``ice edge lat. long.'' (with position in degrees and minutes). This information shall be conveyed in BUFR/CREX via the use of an appropriate 2~05~YYY field as an extra descriptor following the basic template. {[}12.3.7.3{]}

\textbf{B/C10.5.2.4} If the ship is in the open sea reporting an ice edge, the sea-ice concentration (0~20~034) and ice development (0~20~037) shall be reported only if the ship is close to the ice (i.e. within 0.5 nautical mile). {[}12.3.7.4{]}

\textbf{B/C10.5.2.5} If the ship is in an open lead more than 1.0 nautical mile wide, sea-ice concentration (0~20~034) shall be set to 1 and bearing of ice edge (0~20~038) to 0. If the ship is in fast ice with ice boundary beyond limit of visibility, sea-ice concentration (0~20~034) shall be set to 1 and bearing of ice edge (0~20~038) to missing. {[}12.3.7.5{]}

\textbf{B/C10.5.2.6} If no sea ice is visible and the sequence \textless0~20~034, 0~20~035, 0~20~036, 0~20~037, 0~20~038\textgreater{} is used to report ice of land origin only, 0~20~035 shall be used to report the amount of ice of land origin, and 0~20~034 and 0~20~036 shall be set to 0, and 0~20~037 and 0~20~038 shall be set to missing; e.g. \textless0,2,0, missing, missing\textgreater{} would mean 6--10 icebergs in sight, but no sea ice. {[}12.3.7.6{]}

\textbf{B/C10.5.2.7} In coding concentration or arrangement of sea ice (0~20~034) that condition shall be reported which is of the most navigational significance. {[}12.3.7.7{]}

\textbf{B/C10.5.2.8} The bearing of the principal ice edge reported shall be to the closest part of that edge. {[}12.3.7.8{]}

\textbf{B/C10.5.2.9} The requirements for sea-ice reporting are covered in the following way by the associated parameters:

\begin{quote}
\textbf{Sea-ice concentration} -- Code table 0~20~034

(a) The purpose of the code figure 0 in code table 0~20~034 is to establish in relation to code figure 0 in code table 0~20~036 and code table 0~20~035 whether the floating ice that is visible is only ice of land origin;

(b) The possible variation in sea-ice concentration and arrangement within an area of observation are almost infinite. However, the field of reasonably accurate observation from a ship's bridge is limited. For this reason, and also because minor variations are of temporary significance, the choice of concentrations and arrangements has been restricted for reporting purposes to those representing significantly different conditions from a navigational point of view. The code figures 2--9 have been divided into two sections depending on:

(i) Whether sea-ice concentration within the area of observation is more or less uniform (code figures 2--5); or

(ii) Whether there are marked contrasts in concentration or arrangement (code figures 6--9).

\textbf{Amount and type of ice} -- Code table 0~20~035

(a) This code provides a scale of increasing navigational hazard;

(b) Growlers and bergy bits, being much smaller and lower in the water than icebergs, are more difficult to see either by eye or radar. This is especially so if there is heavy sea running. For this reason, code figures 4 and 5 represent more hazardous conditions than code figures 1 to 3.

\textbf{Ice situation} -- Code table 0~20~036

(a) The purpose of this parameter is to establish:

(i) Whether the ship is in pack ice or is viewing floating ice (i.e. sea ice and/or ice of land origin) from the open sea; and

(ii) A qualitative estimate, dependent on the sea-ice navigation capabilities of the reporting ship, of the penetrability of the sea ice and of the recent trend in conditions;

(b) The reporting of the conditions represented by code figures 1--9 in Code table 0~20~036 can be used to help in the interpretation of reports from the two code tables (ice concentration 0~20~034 and ice development 0~20~037).

\textbf{Ice development} -- Code table 0~20~037

(a) This code table represents a series of increasing navigational difficulties for any given concentration; i.e. if the concentration is, for example, 8/10ths, then new ice would hardly have any effect on navigation while predominantly old ice would provide difficult conditions requiring reductions in speed and frequent course alternations;

(b) The correlation between the stage of development of sea ice and its thickness is explained in the \emph{Guide to Meteorological Instruments and Methods of Observation} (WMO-No. 8).

\textbf{Bearing of ice edge} -- 0~20~038

There is no provision in this code for the reporting of distance from the ice edge. It will be assumed by those receiving the report that the bearing has been given to the closest part of the ice edge. From the reported code figures for ice concentration 0~20~034 and ice development 0~20~037, it will be clear whether the ship is in ice or within 0.5 nautical mile of the ice edge. If the ship is in open water and more than 0.5 nautical mile from the ice edge, the ice edge will be assumed to be aligned at right angles to the bearing which is reported.
\end{quote}

\textbf{B/C10.6 Ship marine data \textless3~02~057\textgreater{}}

\textbf{B/C10.6.1 Sea/water temperature \textless3~02~056\textgreater{}}

\begin{quote}
Method of sea/water temperature measurement shall be reported by Code table 0~02~038; depth bellow sea/water surface (0~07~063) shall be reported in metres (with precision in hundredths of a metre). Sea/water temperature (0~22~043) shall be reported in kelvin (with precision in hundredths of a kelvin); if produced in CREX, in degrees Celsius (with precision in hundredths of a degree Celsius). Sea/water temperature data shall be reported with precision in hundredths of a degree even if they are available with the accuracy in tenths of a degree.

Note: Notes 1 and 2 under Regulation B/C10.4.1.2 shall apply.
\end{quote}

\textbf{B/C10.6.1.1} Sea/water temperature shall always be included in reports from ocean weather stations, when data are available. {[}12.3.2{]}

\textbf{B/C10.6.2 Instrumental wave data \textless3~02~021\textgreater{}}

\begin{quote}
Direction of waves (0~22~001) shall be used to reported true direction (direction from which the waves are coming) in degrees true.

Period of waves (0~22~011) shall be reported in seconds.

Height of waves (0~22~021) shall be reported in metres with precision in tenths of a metre.

Note: Height of waves shall be reported with precision in tenths of a metre even if the data are available with lower accuracy and reported in TAC in units of 0.5 metre. {[}12.3.3.2{]}
\end{quote}

\textbf{B/C10.6.2.1} These data shall always be included in reports from ocean weather stations, when data are available. {[}12.3.3.1{]}

\textbf{B/C10.6.2.2} The sequence 3~02~021 shall be used to report instrumental wave data. {[}12.3.3.2{]}

\textbf{B/C10.6.2.3} When the sea is calm (no waves and no swell) direction of waves, period of waves and height of waves shall be reported as 0. {[}12.3.3.4(a){]}, {[}12.3.3.5(a){]}

\textbf{B/C10.6.2.4} If instrumental wave data are not available for direction, period or height of waves, as the case may be, 0~22~001, 0~22~011 or 0~22~021 shall be set to missing. {[}12.3.3.4(c){]}

\textbf{B/C10.6.3 Wind waves and swell waves \textless3~02~024\textgreater{}}

\begin{quote}
Direction of wind waves (0~22~002) shall be used to reported true direction (direction from which the waves are coming) in degrees true.

Period of wind waves (0~22~012) shall be reported in seconds.

Height of wind waves (0~22~022) shall be reported in metres with precision in tenths of a metre.

Direction of swell waves (0~22~003) shall be used to reported true direction (direction from which the waves are coming) in degrees true.

Period of swell waves (0~22~013) shall be reported in seconds.

Height of swell waves (0~22~023) shall be reported in metres with precision in tenths of a metre.
\end{quote}

\textbf{B/C10.6.3.1} Wind wave data and swell wave data shall always be included in reports from ocean weather stations, when data are available. {[}12.3.3.1{]}, {[}12.3.4.4{]}

\textbf{B/C10.6.3.2} The sequence \textless0~22~002, 0~22~012, 0~22~022\textgreater{} shall be used to report wind waves, when instrumental wave data are not available. {[}12.3.3.3{]}

\textbf{B/C10.6.3.3} When the sea is calm (no waves and no swell) direction, period and height of wind waves shall be reported as 0. {[}12.3.3.4(a){]}

\textbf{B/C10.6.3.4} If wind wave data are not available (owing to confused sea or for any other reason) for direction, period or height of wind waves, as the case may be, 0~22~002, 0~22~012 or 0~22~022 shall be set to missing. {[}12.3.3.4(b), (d){]}

\textbf{B/C10.6.3.5} Swell wave data shall be reported only when swell waves can be distinguished from wind waves. {[}12.3.4.1{]}

\textbf{B/C10.6.3.6} When the sea is calm (no waves and no swell) direction, period and height of swell waves shall be reported as 0.

\textbf{B/C10.6.3.7} If swell waves cannot be distinguished from wind waves, direction 0~22~003, period 0~22~013 and height 0~22~023 of swell waves shall be set to missing.

\textbf{B/C10.6.3.8} If only one system of swell is observed, direction, period and height of swell waves shall be reported in the first replication of \textless3~02~023\textgreater{} = \textless0~22~003, 0~22~013, 0~22~023\textgreater. All elements in the second replication of \textless3~02~023\textgreater{} shall be set to missing. {[}12.3.4.2{]}

\textbf{B/C10.6.3.9} If a second system of swell is observed, its direction, period and height shall be reported in the second replication of \textless3~02~023\textgreater{} = \textless0~22~003, 0~22~013, 0~22~023\textgreater. The corresponding data for the first system of swell shall be reported as prescribed by Regulation B/C10.6.3.8. {[}12.3.4.3{]}

\textbf{B/C10.7 ``Instantaneous'' data required by regional or national reporting practices}

\begin{quote}
If regional or national reporting practices require inclusion of additional ``instantaneous'' parameters, the sequence descriptor 3~08~009 shall be supplemented by the required element descriptors being preceded by a relevant time period descriptor set to zero, i.e. 0~04~024 = 0 or 0~04~025 = 0.

Notes:

(1) ``Instantaneous'' parameter is a parameter that is not coupled to a time period descriptor, e.g. 0~04~024, 0~04~025.

(2) No regional requirements are currently indicated for reporting SHIP data from sea stations in the \emph{Manual on Codes} (WMO-No. 306), Volume II.
\end{quote}

\textbf{B/C10.8 Ship ``period'' data \textless3~02~060\textgreater{}}

\textbf{B/C10.8.1 Present and past weather \textless3~02~038\textgreater{}}

\textbf{B/C10.8.1.1} Present weather (Code table 0~20~003) and past weather (1) (Code table 0~20~004) and past weather (2) (Code table 0~20~005) shall be reported as non-missing values if present and past conditions are known. In case of a report from a manually operated station after a period of closure or at start up, when past weather conditions for the period applicable to the report are unknown, past weather (1) and past weather (2) reported as missing shall indicate that previous conditions are unknown. This regulation shall also apply to automatic reporting stations with the facility to report present and past weather. {[}12.2.6.1{]}

\textbf{B/C10.8.1.2} Code figures 0, 1, 2, 3, 100, 101, 102 and 103 for present weather and code figures 0, 1, 2 and 10 for past weather (1) and past weather (2) shall be considered to represent phenomena without significance. {[}12.2.6.2{]}

\textbf{B/C10.8.1.3} Present and past weather shall be \emph{reported if observation was made (data available), regardless significance of the phenomena.}

\begin{quote}
\emph{Note: If data are produced and collected in traditional codes and present weather and past weather is omitted in a SHIP report (no significant phenomena observed), code figure 508 shall be used for present weather and code figure 10 for past weather} (1) and past weather (2) when converted \emph{into BUFR or CREX.}
\end{quote}

\textbf{B/C10.8.1.4} If no observation was made (data not available)\emph{, code figure 509 shall be used for present weather and both past weather} (1) and past weather (2) \emph{shall be reported as missing.}

\textbf{B/C10.8.1.5} \textbf{Present weather from a manned weather station}

\textbf{B/C10.8.1.5.1} If more than one form of weather is observed, the highest applicable code figure from the range \textless00 to 99\textgreater{} shall be selected for present weather. Code figure 17 shall have precedence over code figures 20--49. Other weather may be reported using additional entries 0~20~003 or 0~20~021 to 0~20~026 applying Regulation B/C10.7. {[}12.2.6.4.1{]}

\textbf{B/C10.8.1.5.2} In coding 01, 02, or 03, there is no limitation on the magnitude of the change of the cloud amount. Code figures 00, 01, and 02 can each be used when the sky is clear at the time of observation. In this case, the following interpretation of the specifications shall apply:

\begin{quote}
-- 00 is used when the preceding conditions are not known;

-- 01 is used when the clouds have dissolved during the past hour;

-- 02 is used when the sky has been continuously clear during the past hour.

{[}12.2.6.4.2{]}
\end{quote}

\textbf{B/C10.8.1.5.3} When the phenomenon is not predominantly water droplets, the appropriate code figure shall be selected without regard to visibility. {[}12.2.6.4.3{]}

\textbf{B/C10.8.1.5.4} The code figure 05 shall be used when the obstruction to vision consists predominantly of lithometeors. {[}12.2.6.4.4{]}

\textbf{B/C10.8.1.5.5} National instructions shall be used to indicate the specifications for code figures 07 and 09. {[}12.2.6.4.5{]}

\textbf{B/C10.8.1.5.6} The visibility restrictions on code figure 10 shall be 1~000 metres or more. The specification refers only to water droplets and ice crystals. {[}12.2.6.4.6{]}

\textbf{B/C10.8.1.5.7} For code figures 11 or 12 to be reported, the apparent visibility shall be less than 1~000 metres. {[}12.2.6.4.7{]}

\textbf{B/C10.8.1.5.8} For code figure 18, the following criteria for reporting squalls shall be used:

\begin{quote}
(a) When wind speed is measured: A sudden increase of wind speed of at least eight metres per second, the speed rising to 11 metres per second or more and lasting for at least one minute;

(b) When the Beaufort scale is used for estimating wind speed: A sudden increase of wind speed by at least three stages of the Beaufort scale, the speed rising to force 6 or more and lasting for at least one minute.

{[}12.2.6.4.8{]}
\end{quote}

\textbf{B/C10.8.1.5.9} Code figures 20--29 shall never be used when precipitation is observed at the time of observation. {[}12.2.6.4.9{]}

\textbf{B/C10.8.1.5.10} For code figure 28, visibility shall have been less than 1~000 metres.

\begin{quote}
Note: The specification refers only to visibility restrictions which occurred as a result of water droplets or ice crystals.

{[}12.2.6.4.10{]}
\end{quote}

\textbf{B/C10.8.1.5.11} For synoptic coding purposes, a thunderstorm shall be regarded as being at the station from the time thunder is first heard, whether or not lightning is seen or precipitation is occurring at the station. A thunderstorm shall be reported if thunder is heard within the normal observational period preceding the time of\\
the report. A thunderstorm shall be regarded as having ceased at the time thunder is last heard and the cessation is confirmed if thunder is not heard for 10--15 minutes after this time. {[}12.2.6.4.11{]}

\textbf{B/C10.8.1.5.12} The necessary uniformity in reporting code figures 36, 37, 38, and 39, which may be desirable within certain regions, shall be obtained by means of national instructions. {[}12.2.6.4.12{]}

\textbf{B/C10.8.1.5.13} A visibility restriction «~less than 1~000 metres~» shall be applied to code figures 42--49. In the case of code figures 40 or 41, the apparent visibility in the fog or ice fog patch or bank shall be less than 1~000 metres. Code figures 40--47 shall be used when the obstructions to vision consist predominantly of water droplets or ice crystals, and 48 or 49 when the obstructions consist predominantly of water droplets. {[}12.2.6.4.13{]}

\textbf{B/C10.8.1.5.14} When referring to precipitation, the phrase «~at the station~» in the code table shall mean «~at the point where the observation is normally taken~». {[}12.2.6.4.14{]}

\textbf{B/C10.8.1.5.15} The precipitation shall be encoded as intermittent if it has been discontinuous during the preceding hour, without presenting the character of a shower. {[}12.2.6.4.15{]}

\textbf{B/C10.8.1.5.16} The intensity of precipitation shall be determined by the intensity at the time of the observation. {[}12.2.6.4.16{]}

\textbf{B/C10.8.1.5.17} Code figures 80--90 shall be used only when the precipitation is of the shower type and takes place at the time of the observation.

\begin{quote}
Note: Showers are produced by convective clouds. They are characterized by their abrupt beginning and end and by the generally rapid and sometimes great variations in the intensity of the precipitation. Drops and solid particles falling in a shower are generally larger than those falling in non-showery precipitation. Between showers openings may be observed unless stratiform clouds fill the intervals between the cumuliform clouds.

{[}12.2.6.4.17{]}
\end{quote}

\textbf{B/C10.8.1.5.18} In reporting code figure 98, the observer shall be allowed considerable latitude in determining whether precipitation is or is not occurring, if it is not actually visible. {[}12.2.6.4.18{]}

\textbf{B/C10.8.1.6 Present weather from an automatic weather station}

\textbf{B/C10.8.1.6.1} The highest applicable code figure shall be selected. {[}12.2.6.5.1{]}

\textbf{B/C10.8.1.6.2} In coding code figures 101, 102, and 103, there is no limitation on the magnitude of the change of the cloud amount. Code figures 100, 101, and 102 can each be used when the sky is clear at the time of observation. In this case, the following interpretation of the specifications shall apply:

\begin{quote}
-- Code figure 100 is used when the preceding conditions are not known;

-- Code figure 101 is used when the clouds have dissolved during the past hour;

-- Code figure 102 is used when the sky has been continuously clear during the past hour.

{[}12.2.6.5.2{]}
\end{quote}

\textbf{B/C10.8.1.6.3} When the phenomenon is not predominantly water droplets, the appropriate code figure shall be selected without regard to the visibility. {[}12.2.6.5.3{]}

\textbf{B/C10.8.1.6.4} The code figures 104 and 105 shall be used when the obstruction to vision consists predominantly of lithometeors. {[}12.2.6.5.4{]}

\textbf{B/C10.8.1.6.5} The visibility restriction on code figure 110 shall be 1~000 metres or more. The specification refers only to water droplets and ice crystals. {[}12.2.6.5.5{]}

\textbf{B/C10.8.1.6.6} For code figure 118, the following criteria for reporting squalls shall be used:

\begin{quote}
A sudden increase of wind speed of at least eight metres per second, the speed rising to 11 metres per second or more and lasting for at least one minute.

{[}12.2.6.5.6{]}
\end{quote}

\textbf{B/C10.8.1.6.7} Code figures 120--126 shall never be used when precipitation is observed at the time of observation. {[}12.2.6.5.7{]}

\textbf{B/C10.8.1.6.8} For code figure 120, visibility shall have been less than 1~000 metres.

\begin{quote}
Note: The specification refers only to visibility restrictions, which occurred as a result of water droplets or ice crystals.

{[}12.2.6.5.8{]}
\end{quote}

\textbf{B/C10.8.1.6.9} For synoptic coding purposes, a thunderstorm shall be regarded as being at the station from the time thunder is first detected, whether or not lightning is detected or precipitation is occurring at the station. A thunderstorm shall be reported in present weather if thunder is detected within the normal observational period preceding the time of the report. A thunderstorm shall be regarded as having ceased at the time thunder is last detected and the cessation is confirmed if thunder is not detected for 10--15 minutes after this time. {[}12.2.6.5.9{]}

\textbf{B/C10.8.1.6.10} A visibility restriction «~less than 1~000 metres~» shall be applied to code figures 130--135. {[}12.2.6.5.10{]}

\textbf{B/C10.8.1.6.11} The precipitation shall be encoded as intermittent if it has been discontinuous during the preceding hour, without presenting the character of a shower. {[}12.2.6.5.11{]}

\textbf{B/C10.8.1.6.12} The intensity of precipitation shall be determined by the intensity at the time of observation. {[}12.2.6.5.12{]}

\textbf{B/C10.8.1.6.13} Code figures 180--189 shall be used only when the precipitation is intermittent or of the shower type and takes place at the time of observation.

\begin{quote}
Note: Showers are produced by convective clouds. They are characterized by their abrupt beginning and end and by the generally rapid and sometimes great variations in the intensity of the precipitation. Drops and solid particles falling in a shower are generally larger than those falling in non-showery precipitation. Between showers openings may be observed unless stratiform clouds fill the intervals between the cumuliform clouds.

{[}12.2.6.5.13{]}
\end{quote}

\textbf{B/C10.8.1.7 Past weather reported from a manned weather station}

\textbf{B/C10.8.1.7.1 Time period}

\begin{quote}
The time period (0~04~024) covered by past weather (1) and past weather (2) shall be expressed as \emph{a negative value} in hours:

(a) Six hours, for observations at 0000, 0600, 1200, and 1800 UTC;

(b) Three hours for observations at 0300, 0900, 1500, and 2100 UTC;

(c) Two hours for intermediate observations if taken every two hours;

(d) \emph{One hour for intermediate observations if taken every hour}.

{[}12.2.6.6.1{]}
\end{quote}

\textbf{B/C10.8.1.7.2} The code figures for past weather (1) and past weather (2) shall be selected in such a way that past and present weather together give as complete a description as possible of the weather in the time interval concerned. For example, if the type of weather undergoes a complete change during the time interval concerned, the code figures selected for past weather (1) and past weather (2) shall describe the weather prevailing before the type of weather indicated by present weather began. {[}12.2.6.6.2{]}

\textbf{B/C10.8.1.7.3} When the past weather (1) and past weather (2) are used in hourly reports, Regulation B/C10.8.1.7.1 (d) shall apply. {[}12.2.6.6.3{]}

\textbf{B/C10.8.1.7.4} If, using Regulation B/C10.8.1.7.2, more than one code figure may be given to past weather (1), the highest figure shall be reported for past weather (1) and the second highest code figure shall be reported for past weather (2). {[}12.2.6.6.4{]}

\textbf{B/C10.8.1.7.5} If the weather during the period has not changed so that only one code figure may be selected for past weather, then that code figure shall be reported for both past weather (1) and past weather (2). {[}12.2.6.6.5{]}

\textbf{B/C10.8.1.8 Past weather reported from an automatic weather station}

\textbf{B/C10.8.1.8.1 Time period}

\begin{quote}
The time period (0~04~024) covered by past weather (1) and past weather (2) shall be expressed as \emph{a negative value} in hours:

(a) Six hours for observations at 0000, 0600, 1200, and 1800 UTC;

(b) Three hours for observations at 0300, 0900, 1500, and 2100 UTC;

(c) Two hours for intermediate observations if taken every two hours;

(d) \emph{One hour for intermediate observations if taken every hour}.

{[}12.2.6.7.1{]}
\end{quote}

\textbf{B/C10.8.1.8.2} The code figures for past weather (1) and past weather (2) shall be selected so that the maximum capability of the automatic station to discern past weather is utilized, and so that past and present weather together give as complete a description as possible of the weather in the time interval concerned. {[}12.2.6.7.2{]}

\textbf{B/C10.8.1.8.3} In cases where the automatic station is capable only of discerning very basic weather conditions, the lower code figures representing basic and generic phenomena may be used. If the automatic station has higher discrimination capabilities, the higher code figures representing more detailed explanation of the phenomena shall be used. For each basic type of phenomenon, the highest code figure within the discrimination capability of the automatic station shall be reported. {[}12.2.6.7.3{]}

\textbf{B/C10.8.1.8.4} If the type of weather during the time interval concerned undergoes complete and discernible changes, the code figures selected for past weather (1) and past weather (2) shall describe the weather prevailing before the type of weather indicated by present weather began. The highest figure shall be reported for past weather (1) and the second highest code figure shall be reported for past weather (2). {[}12.2.6.7.4{]}

\textbf{B/C10.8.1.8.5} If a discernible change in weather has not occurred during the period, so that only one code figure may be selected for the past weather, then that code figure shall be reported for both past weather (1) and past weather (2). For example, rain during the entire period shall be reported as code figure 14 for both past weather (1) and past weather (2) in the case of an automatic station incapable of differentiating types of precipitation, or code figure 16 for both past weather (1) and past weather (2) in the case of a station with the higher discrimination capability. {[}12.2.6.7.5{]}

\textbf{B/C10.8.2 Precipitation measurement \textless3~02~040\textgreater{}}

\textbf{B/C10.8.2.1 Height of sensor above marine deck platform}

\begin{quote}
Height of sensor above marine deck platform (0~07~032) for precipitation measurement shall be reported in metres (with precision in hundredths of a metre).

This datum represents the actual height of the rain gauge rim above marine deck platform at the point where the rain gauge is located.
\end{quote}

\textbf{B/C10.8.2.2 Period of reference for amount precipitation}

\begin{quote}
Time period (0~04~024) for amount of precipitation shall be reported as \emph{a negative value} in hours. It shall be determined:

(a) By regional decision (e.g. --6, --12, --24) in the first replication;

(b) By national decision (e.g. --1, --3) in the second replication.
\end{quote}

\textbf{B/C10.8.2.3 Total amount of precipitation}

\begin{quote}
Total amount of precipitation, which has fallen during the period of reference for amount of precipitation, shall be reported in kilograms per square metre (with precision in tenths of a kilogram per square metre).
\end{quote}

\textbf{B/C10.8.2.3.1} Precipitation, when it can be and has to be reported, shall be reported as 0.0~kg~m\textsuperscript{--2} if no precipitation were observed during \textbf{the} referenced period. {[}12.2.5.4{]}

\textbf{B/C10.8.2.3.2} Trace shall be reported as ``--0.1 kg~m\textsuperscript{--2}''.

\textbf{B/C10.8.3 Ship extreme temperature data \textless3~02~058\textgreater{}}

\textbf{B/C10.8.3.1 Height of sensor above marine deck platform and height of sensor above water surface}

\begin{quote}
Height of sensor above marine deck platform (0~07~032) for temperature measurement shall be reported in metres (with precision in hundredths of a metre).

This datum represents the actual height of temperature sensors above marine deck platform at the point where the sensors are located.

Height of sensor above water surface (0~07~033) for temperature measurement shall be reported in metres (with precision in tenths of a metre).

This datum represents the actual height of temperature sensors above marine water surface of sea or lake.
\end{quote}

\textbf{\\
}

\textbf{B/C10.8.3.2 Periods of reference for extreme temperatures}

\begin{quote}
Time period for maximum temperature and time period for minimum temperature (0~04~024) shall be determined by regional decision and reported as \emph{negative values} in hours. {[}12.4.4{]}

Notes:

(1) If the period for maximum temperature or the period for minimum temperature ends at the nominal time of report, the second value of 0~04~024 shall be reported as 0.

(2) If the period for maximum temperature or the period for minimum temperature does not end at the nominal time of report, the first value of 0~04~024 shall indicate the beginning of the period of reference and the second value of 0~04~024 shall indicate the end of the period of reference. E.g. to report the maximum temperature for the previous calendar day from a station in RA IV, value of the first 0~04~024 shall be set to --30 and value of the second 0~04~024 shall be set to --6, provided that the nominal time of the report 12 UTC corresponds to 6 a.m. local time.
\end{quote}

\textbf{B/C10.8.3.3 Maximum and minimum temperature}

\begin{quote}
Maximum and minimum temperature shall be reported in kelvin (with precision in hundredths of a kelvin); if produced in CREX, in degrees Celsius (with precision in hundredths of a degree Celsius).

Note: Notes 1 and 2 under Regulation B/C10.4.1.2 shall apply.
\end{quote}

\textbf{B/C10.8.4 Ship wind data \textless3~02~059\textgreater{}}

\textbf{B/C10.8.4.1 Height of sensor above marine deck platform and height of sensor above water surface}

\begin{quote}
Height of sensor above marine deck platform (0~07~032) for wind measurement shall be reported in metres (with precision in hundredths of a metre).

This datum represents the actual height of wind sensors above marine deck platform at the point where the sensors are located.

Height of sensor above water surface (0~07~033) for wind measurement shall be reported in metres (with precision in tenths of a metre).

This datum represents the actual height of wind sensors above marine water surface of sea or lake.
\end{quote}

\textbf{B/C10.8.4.2 Type of instrumentation for wind measurement} -- Flag table 0~02~002

\begin{quote}
This datum shall be used to specify whether the wind speed was measured by certified instruments (bit No. 1 set to 1) or estimated on the basis of the Beaufort wind scale (bit No. 1 set to 0), and to indicate the original units for wind speed measurement. Bit No. 2 set to 1 indicates that wind speed was originally measured in knots and bit No. 3 set to 1 indicates that wind speed was originally measured in kilometres per hour. Setting both bits No. 2 and No. 3 to 0 indicates that wind speed was originally measured in metres per second.
\end{quote}

\textbf{B/C10.8.4.3 Wind direction} \textbf{and speed}

\begin{quote}
The mean direction and speed of the wind over the 10-minute period immediately preceding the observation shall be reported. The time period (0~04~025) shall be included as --10. However, when the 10-minute period includes a discontinuity in the wind characteristics, only data obtained after the discontinuity shall be used for reporting the mean values, and hence the period (0~04~025) in these circumstances shall be correspondingly reduced. {[}12.2.2.3.1{]}

The time period is preceded by a time significance qualifier (0~08~021) that shall be set to 2 (Time averaged).

The wind direction (0~11~001) shall be reported in degrees true and the wind speed (0~11~002) shall be reported in metres per second (with precision in tenths of a metre per second).

Note: Surface wind direction measured at a station within 1° of the North Pole or within 1° of the South Pole shall be reported in such a way that the azimuth ring shall be aligned with its zero coinciding with the Greenwich 0° meridian.
\end{quote}

\textbf{B/C10.8.4.3.1} In the absence of wind instruments, the wind speed shall be estimated on the basis of the Beaufort wind scale. The Beaufort number obtained by estimation is converted into metres per second by use of the relevant wind speed equivalent column on the Beaufort scale, and this speed is reported for wind speed. {[}12.2.2.3.2{]}

\textbf{B/C10.8.4.3.2} Calm shall be reported by setting wind direction to 0 and wind speed to 0. Variable shall be reported by setting wind direction to 0 and wind speed to a positive \emph{non-missing} value.

\textbf{B/C10.8.4.4 Maximum wind gust direction and speed}

\begin{quote}
Time period for maximum wind gust direction and speed (0~04~025) shall be determined by regional or national decision and reported as a negative value in minutes.

Direction of the maximum wind gust (0~11~043) shall be reported in degrees true and speed of the maximum wind gust (0~11~041) shall be reported in metres per second (with precision in tenths of a metre per second).
\end{quote}

\textbf{B/C10.9 ``Period'' data required by regional or national reporting practices}

\begin{quote}
If regional reporting practices in a Region require inclusion of additional ``period'' parameters, the corresponding ``regional'' common sequence (see the annex to B/C1) shall be supplemented by relevant descriptors. If national reporting practices require inclusion of additional ``period'' parameters, the common sequence 3~08~009 shall be supplemented by relevant descriptors.

Notes:

(1) ``Period'' parameter is a parameter that is coupled to a time period descriptor, e.g. 0~04~024, 0~04~025.

(2) No additional ``period'' parameters are currently required by regional regulations for SHIP data in the \emph{Manual on Codes} (WMO-No. 306), Volume II.
\end{quote}

\_\_\_\_\_\_\_\_\_\_\_\_\_
